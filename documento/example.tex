% Template:     Informe LaTeX
% Documento:    Archivo de ejemplo
% Versión:      8.2.6 (06/07/2023)
% Codificación: UTF-8
%
% Autor: Pablo Pizarro R.
%        pablo@ppizarror.com
%
% Manual template: [https://latex.ppizarror.com/informe]
% Licencia MIT:    [https://opensource.org/licenses/MIT]

% ------------------------------------------------------------------------------
% NUEVA SECCIÓN
% ------------------------------------------------------------------------------
% Las secciones se inician con \section, si se quiere una sección sin número se
% pueden usar las funciones \sectionanum (sección sin número) o la función
% \sectionanumnoi para crear el mismo título sin numerar y sin aparecer en el índice
\section{Herramientas utilizadas}
\subsection{Godot}
        ``Godot Engine es un motor de juegos multiplataforma repleto de funciones para crear
        juegos en 2D y 3D desde una interfaz unificada. Ofrece un completo conjunto de herramientas comunes, para que los usuarios
        puedan centrarse en crear juegos sin tener que reinventar la rueda. Los juegos se pueden exportar con
        un solo clic a varias plataformas, incluidas las principales plataformas de escritorio (Linux, macOS, Windows),
        plataformas móviles (Android, iOS), así como plataformas basadas en web y consolas.'' (Extraido de su pagina en GitHub) \\

        Godot fue el único programa que utilicé para hacer mi proyecto, ya que viene muy bien integrado con todo lo que necesitaba, 
        tiene su propio lenguaje para shaders y scripts, una interfaz muy simple para alguien que está recien aprendiendo,
        pero la verdadera razón por la que escogí a Godot es por su excelente soporte en Linux (sistema operativo que utilizo principalmente). \\

        Fue una experiencia muy grata utilizar este programa aunque si bien hay muchisimos más tutoriales para Unity fue sencillo encontrar todo
        lo que necesitaba.
        
\subsection{The Models Resource}
        The Models Resource fue la pagina web principal de donde conseguía los modelos para mi programa,
        y basicamente consiste en una web donde un grupo de aficionados van recompilando modelos de videojuegos
        tanto antiguos como nuevos y los hacen públicos. \\

        Mi programa solo ha sido probado con modelos de este sitio web.  

\subsection{Firefox}
        Y bueno obviamente utilicé un navegador para buscar recursos, documentación y tutoriales.

\clearpage



\section{Objetivos del proyecto}

% SUB-SECCIÓN
% Las sub-secciones se inician con \subsection, si se quiere una sub-sección
% sin número se pueden usar las funciones \subsectionanum (nuevo subtítulo sin
% numeración) o la función \subsectionanumnoi para crear el mismo subtítulo sin
% numerar y sin aparecer en el índice
\subsection{Objetivos Propuestos}
	\begin{itemize}
	   \item Aprender a utilizar algun programa relacionado a la grafica (Godot o Unity).
	   \item Desarrollar un pipeline que lleve un modelo 3D a Pixel Art de manera simple y sencilla.
	   \item Lograr tambien limitar las paletas de colores de los modelos 3D.
       \item Lograr aplicar paletas de colores limitadas para así emular aún mejor el efecto de consolas antiguas.
	\end{itemize}


\subsection{Objetivos Logrados}
	\begin{itemize}
        \item Aprendí a utilizar de manera comóda Godot 4.
        \item Aprendí como funcionan los shaders de Godot.
        \item Logré hacer un programa que recibe un modelo 3D y lo transforma a Pixel Art.
        \item Logré limitar la paleta de color de cada modelo para lograr un efecto realista.
	\end{itemize}

\subsection{Objetivos No Logrados}
	\begin{itemize}
        \item No logré aplicar paletas de colores personalizadas para cada modelo, pues se me complicó la programación del shader.
	\end{itemize}

\subsection{Adicional}
	\begin{itemize}
        \item Logré aplicar un shader que emula el efecto de las pantallas CRT.
        \item Decoré un poco el programa con una font bonita.
	\end{itemize}
% SUB-SECCIÓN
\clearpage
\section{Pantallazos}

\subsection{Interfaz del programa}

Allí se muestra los efectos que pude lograr y el boton para cargar un modelo 3D.
\insertimage[]{img/elprograma.png}{width=13cm}{Interfaz del programa sin cargar ningun modelo 3D.}

\subsection{Cargar un modelo 3D}
\insertimage[]{img/peranormal.png}{width=13cm}{Programa con un modelo 3D cargado sin efectos.}

\subsection{Efecto Pixel Art}
\insertimage[]{img/perapixelart.png}{width=13cm}{Programa con un modelo 3D con el efecto Pixel Art aplicado.}

\subsection{Efecto Pixel Art + Efecto CRT}
\insertimage[]{img/perapixelartcrt.png}{width=13cm}{Programa con un modelo 3D con el efecto Pixel Art y el efecto de CRT aplicado.}


